\subsection{Display > Toggle Viewer Based Perspective}
\label{subsection:viewerPerspective}

\index{projection!pour visualisation}
\par
Cette commande intervient sur l'affichage interactif dans les contextes graphiques
\index{contexte graphique}en permettant de basculer entre le point de vue centr� sur la cam�ra
et le point de vue\index{point de vue} centr� sur la sc�ne.
\par
Un premier appel � cette commande permet de positionner le centre de rotation du point de vue
sur la cam�ra elle m�me. Ce mode est associ� � une projection perspective\index{projection} (cf. section \ref{subsection:centeredPerspective}).
Lors des phases interactives (mouvement de souris dans un contexte graphique - cf. section \ref{subsection:Interactivit�}),
la cam�ra tournera donc sur elle m�me, en conservant sa position.\\
\par
Si elle est sollicit�e une seconde fois, cette commande r�tablit l'affichage par d�faut centr�
sur la sc�ne et bas� sur la projection orthographique.\\
\par
\textcolor[rgb]{1.00,0.00,0.00}{Raccourci clavier : F4}
